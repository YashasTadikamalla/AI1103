\documentclass[journal,12pt,twocolumn]{IEEEtran}

\usepackage{setspace}
\usepackage{gensymb}
\singlespacing
\usepackage[cmex10]{amsmath}
\usepackage{amssymb}
\usepackage{xurl}
\usepackage{tabularx}
\usepackage{amsthm}
\usepackage{comment}
\usepackage{mathrsfs}
\usepackage{txfonts}
\usepackage{stfloats}
\usepackage{bm}
\usepackage{cite}
\usepackage{cases}
\usepackage{subfig}

\usepackage{longtable}
\usepackage{multirow}

\usepackage{enumitem}
\usepackage{mathtools}
\usepackage{steinmetz}
\usepackage{tikz}
\usepackage{circuitikz}
\usepackage{verbatim}
\usepackage{tfrupee}
\usepackage[breaklinks=true]{hyperref}
\usepackage{graphicx}
\usepackage{tkz-euclide}

\usetikzlibrary{calc,math}
\usepackage{listings}
    \usepackage{color}                                            %%
    \usepackage{array}                                            %%
    \usepackage{longtable}                                        %%
    \usepackage{calc}                                             %%
    \usepackage{multirow}                                         %%
    \usepackage{hhline}                                           %%
    \usepackage{ifthen}                                           %%
    \usepackage{lscape}     
\usepackage{multicol}
\usepackage{chngcntr}

\DeclareMathOperator*{\Res}{Res}

\renewcommand\thesection{\arabic{section}}
\renewcommand\thesubsection{\thesection.\arabic{subsection}}
\renewcommand\thesubsubsection{\thesubsection.\arabic{subsubsection}}

\renewcommand\thesectiondis{\arabic{section}}
\renewcommand\thesubsectiondis{\thesectiondis.\arabic{subsection}}
\renewcommand\thesubsubsectiondis{\thesubsectiondis.\arabic{subsubsection}}


\hyphenation{op-tical net-works semi-conduc-tor}
\def\inputGnumericTable{}                                 %%

\lstset{
%language=C,
frame=single, 
breaklines=true,
columns=fullflexible
}
\begin{document}


\newtheorem{theorem}{Theorem}[section]
\newtheorem{problem}{Problem}
\newtheorem{proposition}{Proposition}[section]
\newtheorem{lemma}{Lemma}[section]
\newtheorem{corollary}[theorem]{Corollary}
\newtheorem{example}{Example}[section]
\newtheorem{definition}[problem]{Definition}

\newcommand{\BEQA}{\begin{eqnarray}}
\newcommand{\EEQA}{\end{eqnarray}}
\newcommand{\define}{\stackrel{\triangle}{=}}
\bibliographystyle{IEEEtran}
\raggedbottom
\setlength{\parindent}{0pt}
\providecommand{\mbf}{\mathbf}
\providecommand{\pr}[1]{\ensuremath{\Pr\left(#1\right)}}
\providecommand{\qfunc}[1]{\ensuremath{Q\left(#1\right)}}
\providecommand{\sbrak}[1]{\ensuremath{{}\left[#1\right]}}
\providecommand{\lsbrak}[1]{\ensuremath{{}\left[#1\right.}}
\providecommand{\rsbrak}[1]{\ensuremath{{}\left.#1\right]}}
\providecommand{\brak}[1]{\ensuremath{\left(#1\right)}}
\providecommand{\lbrak}[1]{\ensuremath{\left(#1\right.}}
\providecommand{\rbrak}[1]{\ensuremath{\left.#1\right)}}
\providecommand{\cbrak}[1]{\ensuremath{\left\{#1\right\}}}
\providecommand{\lcbrak}[1]{\ensuremath{\left\{#1\right.}}
\providecommand{\rcbrak}[1]{\ensuremath{\left.#1\right\}}}
\theoremstyle{remark}
\newtheorem{rem}{Remark}
\newcommand{\sgn}{\mathop{\mathrm{sgn}}}
\providecommand{\abs}[1]{\vert#1\vert}
\providecommand{\res}[1]{\Res\displaylimits_{#1}} 
\providecommand{\norm}[1]{\lVert#1\rVert}
%\providecommand{\norm}[1]{\lVert#1\rVert}
\providecommand{\mtx}[1]{\mathbf{#1}}
\providecommand{\mean}[1]{E[ #1 ]}
\providecommand{\fourier}{\overset{\mathcal{F}}{ \rightleftharpoons}}
%\providecommand{\hilbert}{\overset{\mathcal{H}}{ \rightleftharpoons}}
\providecommand{\system}{\overset{\mathcal{H}}{ \longleftrightarrow}}
	%\newcommand{\solution}[2]{\textbf{Solution:}{#1}}
\newcommand{\solution}{\noindent \textbf{Solution: }}
\newcommand{\cosec}{\,\text{cosec}\,}
\providecommand{\dec}[2]{\ensuremath{\overset{#1}{\underset{#2}{\gtrless}}}}
\newcommand{\myvec}[1]{\ensuremath{\begin{pmatrix}#1\end{pmatrix}}}
\newcommand{\mydet}[1]{\ensuremath{\begin{vmatrix}#1\end{vmatrix}}}
\newcommand*{\permcomb}[4][0mu]{{{}^{#3}\mkern#1#2_{#4}}}
\newcommand*{\perm}[1][-3mu]{\permcomb[#1]{P}}
\newcommand*{\comb}[1][-1mu]{\permcomb[#1]{C}}
\numberwithin{equation}{subsection}
\makeatletter
\@addtoreset{figure}{problem}
\makeatother
\let\StandardTheFigure\thefigure
\let\vec\mathbf
\renewcommand{\thefigure}{\theproblem}
\def\putbox#1#2#3{\makebox[0in][l]{\makebox[#1][l]{}\raisebox{\baselineskip}[0in][0in]{\raisebox{#2}[0in][0in]{#3}}}}
     \def\rightbox#1{\makebox[0in][r]{#1}}
     \def\centbox#1{\makebox[0in]{#1}}
     \def\topbox#1{\raisebox{-\baselineskip}[0in][0in]{#1}}
     \def\midbox#1{\raisebox{-0.5\baselineskip}[0in][0in]{#1}}
\vspace{3cm}
\title{AI1103 : Assignment 4}
\author{Yashas Tadikamalla - AI20BTECH11027}
\maketitle
\newpage
\bigskip
\renewcommand{\thefigure}{\arabic{figure}}
\renewcommand{\thetable}{\arabic{table}}
Download all python codes from 
\begin{lstlisting}
https://github.com/YashasTadikamalla/AI1103/tree/main/Assignment4/codes
\end{lstlisting}
%
and latex codes from 
%
\begin{lstlisting}
https://github.com/YashasTadikamalla/AI1103/blob/main/Assignment4/Assignment4.tex
\end{lstlisting}
\section*{GATE-2015-MA-Problem(32)}
Let $X_{1},X_{2},\dots$, be a sequence of independent and identically distributed random variables with $P(X_{1}=1)=\dfrac{1}{4}$ and $P(X_{1}=2)=\dfrac{3}{4}$. If $\bar X_{n}=\dfrac{1}{n}\displaystyle\sum_{i=1}^{n}X_{i}$,  for $n=1,2,\dots$, then $\displaystyle\lim_{n\to\infty}P(\bar X_{n} \leq 1.8)$ is equal to
\section*{GATE-2015-MA-Solution(32)}
Given,
\begin{align}
\tag{32.1}
    Pr(X_{1}=1)=\dfrac{1}{4},Pr(X_{2}=2)=\dfrac{3}{4}
\end{align}
As $X_{1},X_{2},\dots$, are identically distributed random variables, $\forall i \in \{1,2,\dots,n\}$
\begin{align}
\tag{32.2}
    Pr(X_{i}=1)=\dfrac{1}{4},Pr(X_{i}=2)=\dfrac{3}{4}
\end{align}
Also,
\begin{align}
\tag{32.3}
    \because P(X_{i}&=1)+P(X_{i}=2)=1\\
\tag{32.4}
    &\therefore X_{i} \in \{1,2\}
\end{align}
Let $X \in \{1,2\}$ be the random variable denoting the value of $X_{i}, \forall i \in \{1,2,\dots,n\}$. Let
\begin{align}
\tag{32.5}
    n(&X=1)=n-k,n(X=2)=k\\
\tag{32.6}
    &\Rightarrow\bar X_{n}=\dfrac{1}{n}\displaystyle\sum_{i=1}^{n}X_{i}=\dfrac{n+k}{n}\\
\tag{32.7}
    &\hspace{0.25cm}\therefore\bar X_{n} \leq 1.8\Rightarrow k \leq 0.8n
\end{align}
The cumulative distribution function (CDF) for a binomial distribution is given by \begin{align}
\tag{32.8}
    F_{X}(r)=Pr(X\leq r)=\displaystyle\sum_{k=0}^{r}{\comb{n}{k}}p^{k}q^{n-k}\\
\tag{32.9}
    \therefore P(\bar X_{n} \leq 1.8)=\displaystyle\sum_{k=0}^{0.8n}{\comb{n}{k}}p^{k}q^{n-k}
\end{align}
where, 
\begin{align}
\tag{32.10}
    p=\dfrac{3}{4},q=1-p=\dfrac{1}{4}
\end{align}
\begin{align}
\tag{32.11}
    \Rightarrow\displaystyle\lim_{n\to\infty}P(\bar X_{n} \leq 1.8)=\displaystyle\lim_{n\to\infty}\displaystyle\sum_{k=0}^{0.8n}\dfrac{{\comb{n}{k}}3^{k}}{4^{n}}
\label{eq:val}
\end{align}
On solving \eqref{eq:val}, we get
\begin{align}
\tag{32.12}
    \displaystyle\lim_{n\to\infty}P(\bar X_{n} \leq 1.8)=1
\end{align}





\end{document}